% !TEX TS-program = pdflatex
% !TEX root = ../tesi.tex
%************************************************
\chapter*{Introduzione}
\addcontentsline{toc}{chapter}{INTRODUZIONE}
\label{chp:Introduzione}
%************************************************

Quando ci troviamo in un determinato luogo, che sia un appartamento, un ufficio
o altro, non sempre facciamo caso alle sue proprietà acustiche, magari scorgiamo
altri dettagli, come una certa corrispondenza tra i colori delle pareti e gli
oggetti di arredamento, ma, le relazioni che intercorrono tra l’ambiente e la
percezione sonora, sono informazioni che spesso trascuriamo o che addirittura
risultano superflue.

In un ambiente reale, i rapporti tra gli elementi presenti al suo interno, sono
innumerevoli, definendo in maniera quasi assoluta la peculiarità dei suoni che
si propagano. Possiamo pressocchè dire che l’evento acustico è legato
indissolubilmente allo spazio che ha attorno.

In questo lavoro di tesi affronto argomenti di riverberazione artificiale e
come l'ambiente riverberante influisce sulla percezione sonora. In particolare
la mia ricerca si è concentrata sullo studio del comportamento delle vibrazioni
acustiche al variare delle \emph{condizioni atmosferiche}, in particolare nelle
qualità di: temperatura, umidità e pressione atmosferica; in presenza di mezzi
di trasmissione acustica diversi dall’aria. Lo studio dei fenomeni fisici è
confluito nell'implementazione di un riverbero a parametri di controllo
atmosferici.

La tesi attinge alle ricerche princicpali che nel novecento hanno introdotto il
concetto di riverberazione (Sabine), tecniche di riverberazione artificiale
(Schroeder) e poi elaborato quelle tecniche (Moorer) contribuendo allo sviluppo
delle possibilità di simulazione di un ambiente sonoro realistico.

Il testo è diviso in tre parti: la presente introduzione, una parte storica, le
strategie di implementazione che ho adottato.

All'interno della parte storica mi sono occupato di raccogliere informazioni e
organizzarle, cercando di tracciare un filo conduttore in grado di ripercorrere
i vari stadi dello studio dei riverberi. Partendo dagli studi di \ws, fondamenta
della fisica acustica, ho poi esposto brevemente alcune nozioni essenziali, di
carattere fisico e matematico, per l'implementazione del riverbero, concludendo
infine esplorando l'ambito dei riverberi sintetici, concentrandomi sull'analisi
degli scritti di \ms~ e \jam.

All'interno della parte implementativa tutto ciò che è stato esposto
precedentemente viene riutilizzato in favore della realizzazione del riverbero
\atmoverb. I codici presenti sono scritti in linguaggio \faust~ e ripercorrono
il lavoro eseguito durante il secolo scorso da \ms~ e \jam.

Lo sviluppo di questa tesi è stato guidato principalmente dalla curiosità
dell’autore di comprendere se e quanto, queste caratteristiche interne all’evento
sonoro, possano definirne il timbro e la percezione.

% \todo[inline]{esterne all'evento sonoro? se la tua tesi si basa su elementi che
% modificano la propagazione delle vibrazioni, e se hai compreso che il suono è
% la tua percezione di quelle vibrazioni, come fai a dire che sono esterne… sono
% interne, internissime al suono. :(}
