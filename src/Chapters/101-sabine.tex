\section{Wallace C. Sabine: l'alba dell'acustica architettonica}

Nonostante l'acustica sia stata per millenni presa in considerazione durante il
processo architettonico, la \emph{fisica acustica} ha ottenuto una base
scientifica solida solo nei primi del novecento, grazie agli studi di
\ws\footcite{ws:rev}. Dopo Sabine il tempo di riverbero può essere descritto,
misurato, previsto. Tutte le conosenze attuali, anche l'opera di
\ms\footcite{ms:rev62, ms:rev64}, sono frutto dei suoi studi

\begin{quote}
  The following investigation was not undertaken at first by choice, but devolved
  on the writer in 1895, through instructions from the Corporation of Harvard
  University to propose changes from remedying the acoustical difficulties in
  the lecture-room o the Fogg Art Museum, a building that had just been completed.
  About two years were spent in experimenting on this room, and permanent changes
  where then made. Almost immediately afterward it become certain that a new
  Boston Music Hall would be erected, and the questions arising in the
  consideration of its plans forced a not unwelcome continuance of the general
  investigation\footcite{ws:rev}.
\end{quote}

Curioso notare che tutto il suo lavoro nasca dalla necessità di dover correggere
l'acustica di una sala universitaria adibita a \emph{lecture room}.
Spinto da una mancanza di informazioni e studi pregressi, \ms~ ha costruito una
ricerca organica, con seri problemi da risolvere.

\begin{quote}
  In order that hearing may be good in any auditorium, it is necessary that the
  sound should be sufficiently loud; that the simultaneous components of a
  complex sound should maintain their proper relative intensities;
  and that the successive sounds in rapidly moving articulation, either of speech
  or music, should be clear and distinct, free from each other and from extraneous
  noises. Thesethree are the necessary, as they are the entirely sufficient,
  conditions for good hearing\footcite{ws:rev}.
\end{quote}

Una tripletta di problemi minimi da comprendere e risolvere per rendere accettabile
il riverbero acustico di un ambiente.

\subsection{Loudness}

\ws~ introduce il concetto di propagazione del suono in forma emisferica, che
si riduce proporzionalmente all'aumentare della distanza. Anche con l'aumento
del pubblico, che occupa dunque maggior spazio nella stanza, il suono perde
intensità più rapidamente, assorbito. La parte superiore della propagazione si
muove libera, non affetta da assorbimenti. I primi accorgimenti: elevare
l'oratore ed alzare da terra le file posteriori. Questa soluzione rimanda
inequivocabilmente ad una forma molto conosciuta: il teatro Greco. Un tetto a
coprire la struttura incrementerebbe l'intensità media, soprattutto dei suoni
sostenuti nel tempo, e ne bilancerebbe la resa tra fronte e fondo sala.

\begin{quote}
  The problem of calculating the loudness at different parts of such an
  auditorium is, obviously, complex, but it is perfectly determinate, and as
  soon as the reflecting and absorbing power of the audience and of the various
  wall-surfaces are known it can be solved approximately\footcite{ws:rev}.
\end{quote}

Per la prima volta non parliamo di riverbero, al singolare, ma molti per ogni
ambiente che descriviamo.

\subsection{Interferenze e Risonanze}

Avendo suoni diretti e riflessi che viaggiano nello stesso ambiente, ci si può
imbattere in somme di ampiezza ma anche in cancellazioni se le fasi sono
rispettivamente congruenti o inverse. Tutto questo accade in relazione al suono
emesso, alla sua altezza, che variando, varia l'intero stato di equilibrio,
l'inntero stato di interferenza.

C'è un altro fenomeno che occorre in queste circostanze, in relazione con
l'intererenza, ovvero la risonanza.

\begin{quote}
  The word \emph{resonance} has been used loosely as synonymous with
  \emph{reverberation}, and even with \emph{echo}, and is so given in some of
  the more voluminous but less exact popular dictionaries. In scientific
  literature the term has received a very definite and precise application to
  the phenomenon, wherever it may occur, of the growth of a vibratory motion of
  an elastic body under periodic force stimed to its natural rates of vibration.
  A word having this significance is necessary; and it is very desirable that
  the term should not, even popularly, by meaning many things, cease to mean
  anything exactly\footcite{ws:rev}.
\end{quote}

\subsection{Riverberazione}

Il fenomeno definito riverbero, il processo delle rilessioni multiple, tra le
superfici di un luogo, è alla base della mal comprensione presente nel luogo di
studio. Il riverbero consiste inoltre in una massa di suono che riempie uno
spazio della quale è impossibile cogliere ed analizzare la singola riflessione
e la cui durata. La misurazione temporale diventa quindi fondamentale, oggi
piuttosto scontata per misurazioni fisiche di ordine infinitamente piccole, ma
per \ws~ non era proprio così.

Il percorso di misurazione del tempo di decadimento del \emph{suono residuo}
evidenzia a \ws~che ci sono due e due variabili soltanto di un luogo  ad
influire sul risultato cronometrico: la forma della stanza, inclusa la
dimensione; i materiali, incluso l'arredamento.

Il culmine di questo studio, oltre a dimostrare che esiste una correlazione tra
la quantità di superficie assorbente (pareti, sedute, persone) e la qualità di
percepimento del suono in una stanza, è lo sviluppo di una formula in grado di
ricavare il tempo in cui il suono decade fino ad una situazione di equilibrio.

Parliamo di \emph{RT60} ovvero il tempo in cui il suono (riverberato) decade di
$60 dB$.

\begin{equation}
RT60 = \frac{24(\ln{10})V}{c s_a}
\end{equation}

Di cui:
\begin{compactitem}
\item $V$ è il volume della stanza
\item $c$ è la velocità del suono
\item $s_a$ è il valore di assorbimento totale espresso in Sabins
\end{compactitem}

Possiamo calcolare i \emph{Sabins} sommando l’area totale delle pareti (per
esempio 4 mura + 1 soffitto e 1 pavimento) e moltiplicandola per il coefficiente
di assorbimento (ovviamente il coefficiente può essere diversificato per i
diversi materiali delle pareti). Da notare che il coefficiente è un valore che
varia tra 0 (minimo assorbimento) e 1 (massimo assorbimento).

L’articolo è considerato un capolavoro di acustica applicata e ha avuto una
grande influenza sullo sviluppo della scienza del suono e sulla progettazione
degli spazi sonori.

\subsection{Echi di Sabine}

Ci sono innumerevoli spunti di riflessione tra le pagine dei testi di \ws\footcite{ws:rev},
dai quali, agli scopi di una corretta implementazione digitale del riverbero e
soprattutto agli scopi di un corretto utilizzo musicale, possiamo ricavare:

\begin{compactitem}
  \item La durata del suono residuo ascoltabile in un ambiente è approssimativamente
  uguale in ogni punto dello spazio.
  \item La durata del suono residuo ascoltabile in un ambiente è approssimativamente
  indipendente dalla posizione della sorgente.
\end{compactitem}

Sono questi due presupposti fondamentali, sui quali cercheremo di costruire un
pensiero musicale prima ancora che uno strumento musicale, quale il riverbero
digitale può essere.
