% !TEX TS-program = pdflatex
% !TEX root = ../tesi.tex

%************************************************
\chapter{Conclusioni}
\label{chp:Conclusioni}
%************************************************
Sulla base dei risultati ottenuti nel capitolo \ref{chp:Implementazione}, possiamo trarre alcune
considerazioni sul lavoro svolto.
Si può affermare senza dubbio che le condizioni atmosferiche hanno una certa
influenza sul risultato riverberante: sia il tempo di riverberazione,
che il carattere dell'onda sonora, presentano un
risultato differente in base alle modulazioni parametriche.
Utilizzando temperature e pressioni elevate, il suono riverberante tende ad accorciarsi, 
mostrando il carattere metallico del riverbero.
Inversamente, utilizzando valori sempre più bassi, il riverbero presenta tempi distesi e prolungati 
soprattutto se si aumentano le dimensioni della stanza.
Dall' incremento o diminzuone del valore dei parametri, dunque, è possibile creare effetti 
di cancellazione o enfatizzazione spettrale, che possono indurre ad una non realisticità 
del riverbero. Il suono ``metallizzato'', e una ripetitività delle ritmiche resta comunque
percepibile nonostante gli accorgimenti presi.

\bigskip

A fronte dei risultati e in generale dell'esperienza ottenuta 
svolgendo questa tesi, posso però affermare di trovarmi di fronte a un buon punto di partenza per ulteriori
miglioramenti e approcci diversificati per raggiungere l'obiettivo, e che ciò possa
permettere di relazionarsi in modo diverso con lo strumento riverberante.

% \todo[inline]{credo tu abbia confuso il ruolo del paragrafo conclusivo. non si
% tratta di dare giudizi soggettivi o considerazioni personali. Si trtta di riprendere
% gli argomenti aperti con l'introduzione e concludere il cerchio spiegando cosa
% ha porta a cosa e con quali caratteristiche e corrispondenze. Per esempio:

% il presupposto di controllare un riverbero artificiale ha dato risultati difficilmente
% ottenibili con i parametri tradizionali?

% il controllo parametrico ambientale è qualcosa che fa relazionare con lo strumento
% riverbero in modo diverso dai parametri tradizionali? perchè?
